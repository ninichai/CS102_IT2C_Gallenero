% Options for packages loaded elsewhere
\PassOptionsToPackage{unicode}{hyperref}
\PassOptionsToPackage{hyphens}{url}
%
\documentclass[
]{article}
\usepackage{amsmath,amssymb}
\usepackage{iftex}
\ifPDFTeX
  \usepackage[T1]{fontenc}
  \usepackage[utf8]{inputenc}
  \usepackage{textcomp} % provide euro and other symbols
\else % if luatex or xetex
  \usepackage{unicode-math} % this also loads fontspec
  \defaultfontfeatures{Scale=MatchLowercase}
  \defaultfontfeatures[\rmfamily]{Ligatures=TeX,Scale=1}
\fi
\usepackage{lmodern}
\ifPDFTeX\else
  % xetex/luatex font selection
\fi
% Use upquote if available, for straight quotes in verbatim environments
\IfFileExists{upquote.sty}{\usepackage{upquote}}{}
\IfFileExists{microtype.sty}{% use microtype if available
  \usepackage[]{microtype}
  \UseMicrotypeSet[protrusion]{basicmath} % disable protrusion for tt fonts
}{}
\makeatletter
\@ifundefined{KOMAClassName}{% if non-KOMA class
  \IfFileExists{parskip.sty}{%
    \usepackage{parskip}
  }{% else
    \setlength{\parindent}{0pt}
    \setlength{\parskip}{6pt plus 2pt minus 1pt}}
}{% if KOMA class
  \KOMAoptions{parskip=half}}
\makeatother
\usepackage{xcolor}
\usepackage[margin=1in]{geometry}
\usepackage{color}
\usepackage{fancyvrb}
\newcommand{\VerbBar}{|}
\newcommand{\VERB}{\Verb[commandchars=\\\{\}]}
\DefineVerbatimEnvironment{Highlighting}{Verbatim}{commandchars=\\\{\}}
% Add ',fontsize=\small' for more characters per line
\usepackage{framed}
\definecolor{shadecolor}{RGB}{248,248,248}
\newenvironment{Shaded}{\begin{snugshade}}{\end{snugshade}}
\newcommand{\AlertTok}[1]{\textcolor[rgb]{0.94,0.16,0.16}{#1}}
\newcommand{\AnnotationTok}[1]{\textcolor[rgb]{0.56,0.35,0.01}{\textbf{\textit{#1}}}}
\newcommand{\AttributeTok}[1]{\textcolor[rgb]{0.13,0.29,0.53}{#1}}
\newcommand{\BaseNTok}[1]{\textcolor[rgb]{0.00,0.00,0.81}{#1}}
\newcommand{\BuiltInTok}[1]{#1}
\newcommand{\CharTok}[1]{\textcolor[rgb]{0.31,0.60,0.02}{#1}}
\newcommand{\CommentTok}[1]{\textcolor[rgb]{0.56,0.35,0.01}{\textit{#1}}}
\newcommand{\CommentVarTok}[1]{\textcolor[rgb]{0.56,0.35,0.01}{\textbf{\textit{#1}}}}
\newcommand{\ConstantTok}[1]{\textcolor[rgb]{0.56,0.35,0.01}{#1}}
\newcommand{\ControlFlowTok}[1]{\textcolor[rgb]{0.13,0.29,0.53}{\textbf{#1}}}
\newcommand{\DataTypeTok}[1]{\textcolor[rgb]{0.13,0.29,0.53}{#1}}
\newcommand{\DecValTok}[1]{\textcolor[rgb]{0.00,0.00,0.81}{#1}}
\newcommand{\DocumentationTok}[1]{\textcolor[rgb]{0.56,0.35,0.01}{\textbf{\textit{#1}}}}
\newcommand{\ErrorTok}[1]{\textcolor[rgb]{0.64,0.00,0.00}{\textbf{#1}}}
\newcommand{\ExtensionTok}[1]{#1}
\newcommand{\FloatTok}[1]{\textcolor[rgb]{0.00,0.00,0.81}{#1}}
\newcommand{\FunctionTok}[1]{\textcolor[rgb]{0.13,0.29,0.53}{\textbf{#1}}}
\newcommand{\ImportTok}[1]{#1}
\newcommand{\InformationTok}[1]{\textcolor[rgb]{0.56,0.35,0.01}{\textbf{\textit{#1}}}}
\newcommand{\KeywordTok}[1]{\textcolor[rgb]{0.13,0.29,0.53}{\textbf{#1}}}
\newcommand{\NormalTok}[1]{#1}
\newcommand{\OperatorTok}[1]{\textcolor[rgb]{0.81,0.36,0.00}{\textbf{#1}}}
\newcommand{\OtherTok}[1]{\textcolor[rgb]{0.56,0.35,0.01}{#1}}
\newcommand{\PreprocessorTok}[1]{\textcolor[rgb]{0.56,0.35,0.01}{\textit{#1}}}
\newcommand{\RegionMarkerTok}[1]{#1}
\newcommand{\SpecialCharTok}[1]{\textcolor[rgb]{0.81,0.36,0.00}{\textbf{#1}}}
\newcommand{\SpecialStringTok}[1]{\textcolor[rgb]{0.31,0.60,0.02}{#1}}
\newcommand{\StringTok}[1]{\textcolor[rgb]{0.31,0.60,0.02}{#1}}
\newcommand{\VariableTok}[1]{\textcolor[rgb]{0.00,0.00,0.00}{#1}}
\newcommand{\VerbatimStringTok}[1]{\textcolor[rgb]{0.31,0.60,0.02}{#1}}
\newcommand{\WarningTok}[1]{\textcolor[rgb]{0.56,0.35,0.01}{\textbf{\textit{#1}}}}
\usepackage{graphicx}
\makeatletter
\def\maxwidth{\ifdim\Gin@nat@width>\linewidth\linewidth\else\Gin@nat@width\fi}
\def\maxheight{\ifdim\Gin@nat@height>\textheight\textheight\else\Gin@nat@height\fi}
\makeatother
% Scale images if necessary, so that they will not overflow the page
% margins by default, and it is still possible to overwrite the defaults
% using explicit options in \includegraphics[width, height, ...]{}
\setkeys{Gin}{width=\maxwidth,height=\maxheight,keepaspectratio}
% Set default figure placement to htbp
\makeatletter
\def\fps@figure{htbp}
\makeatother
\setlength{\emergencystretch}{3em} % prevent overfull lines
\providecommand{\tightlist}{%
  \setlength{\itemsep}{0pt}\setlength{\parskip}{0pt}}
\setcounter{secnumdepth}{-\maxdimen} % remove section numbering
\ifLuaTeX
  \usepackage{selnolig}  % disable illegal ligatures
\fi
\IfFileExists{bookmark.sty}{\usepackage{bookmark}}{\usepackage{hyperref}}
\IfFileExists{xurl.sty}{\usepackage{xurl}}{} % add URL line breaks if available
\urlstyle{same}
\hypersetup{
  pdftitle={Pract\_Exam\_Gallenero},
  hidelinks,
  pdfcreator={LaTeX via pandoc}}

\title{Pract\_Exam\_Gallenero}
\author{}
\date{\vspace{-2.5em}2024-03-06}

\begin{document}
\maketitle

\#A Load the built-warpbreaks dataset

\begin{Shaded}
\begin{Highlighting}[]
\CommentTok{\#Load the warpbreaks dataset}
\FunctionTok{data}\NormalTok{ (}\StringTok{"warpbreaks"}\NormalTok{)}
\end{Highlighting}
\end{Shaded}

\#1. Find out, in a single command, which columns of warpbreaks are
either numeric or integer.

\begin{Shaded}
\begin{Highlighting}[]
\NormalTok{Numeric\_cols }\OtherTok{\textless{}{-}} \FunctionTok{sapply}\NormalTok{(warpbreaks, is.numeric)}
\NormalTok{Numeric\_cols}
\end{Highlighting}
\end{Shaded}

\begin{verbatim}
##  breaks    wool tension 
##    TRUE   FALSE   FALSE
\end{verbatim}

\begin{Shaded}
\begin{Highlighting}[]
\CommentTok{\#2. Is numeric a natural data type for the columns which are stored as such? Convert tointeger when necessary.}
\NormalTok{Integer\_cols }\OtherTok{\textless{}{-}} \FunctionTok{sapply}\NormalTok{(warpbreaks, is.integer)}
\NormalTok{Integer\_cols}
\end{Highlighting}
\end{Shaded}

\begin{verbatim}
##  breaks    wool tension 
##   FALSE   FALSE   FALSE
\end{verbatim}

\begin{Shaded}
\begin{Highlighting}[]
\NormalTok{numeric\_or\_integer\_cols }\OtherTok{\textless{}{-}}\NormalTok{ warpbreaks[, Numeric\_cols}\SpecialCharTok{|}\NormalTok{ Integer\_cols]}
\NormalTok{numeric\_or\_integer\_cols}
\end{Highlighting}
\end{Shaded}

\begin{verbatim}
##  [1] 26 30 54 25 70 52 51 26 67 18 21 29 17 12 18 35 30 36 36 21 24 18 10 43 28
## [26] 15 26 27 14 29 19 29 31 41 20 44 42 26 19 16 39 28 21 39 29 20 21 24 17 13
## [51] 15 15 16 28
\end{verbatim}

\#Error messages in R sometimes report the underlying type of an object
rather than theuser-level class. Derive from the following code and
error message what theunderlying type.

for (i in 1:ncol(numeric\_or\_integer\_columns)) \{ if
(!is.integer(numeric\_or\_integer\_columns{[}, i{]})) \{
numeric\_or\_integer\_columns{[}, i{]} \textless-
as.integer(numeric\_or\_integer\_columns{[}, i{]}) \} \}

\begin{Shaded}
\begin{Highlighting}[]
\CommentTok{\#4 ERROR MESSAGE}
\CommentTok{\#Error in 1:ncol(numeric\_or\_integer\_columns) : argument of length 0}
\end{Highlighting}
\end{Shaded}

\#B. Load the example.File.txt \#B

\begin{Shaded}
\begin{Highlighting}[]
\CommentTok{\#Read the complete file using readLines.}
\NormalTok{lines }\OtherTok{\textless{}{-}} \FunctionTok{readLines}\NormalTok{(}\StringTok{"exampleFile.txt"}\NormalTok{)}
\end{Highlighting}
\end{Shaded}

\begin{verbatim}
## Warning in readLines("exampleFile.txt"): incomplete final line found on
## 'exampleFile.txt'
\end{verbatim}

\begin{Shaded}
\begin{Highlighting}[]
\CommentTok{\#Separate the vector of lines into a vector containing comments and a vector containing the data. Hint: use grepl.}
\NormalTok{comments }\OtherTok{\textless{}{-}}\NormalTok{ lines[}\FunctionTok{grepl}\NormalTok{(}\StringTok{"\^{}//"}\NormalTok{, lines)]}
\NormalTok{comments}
\end{Highlighting}
\end{Shaded}

\begin{verbatim}
## [1] "// Survey data. Created : 21 May 2013"
## [2] "// Field 1: Gender"                   
## [3] "// Field 2: Age (in years)"           
## [4] "// Field 3: Weight (in kg)"
\end{verbatim}

\begin{Shaded}
\begin{Highlighting}[]
\NormalTok{data\_lines }\OtherTok{\textless{}{-}}\NormalTok{ lines[}\SpecialCharTok{!}\FunctionTok{grepl}\NormalTok{(}\StringTok{"\^{}//"}\NormalTok{, lines)]}
\NormalTok{data\_lines}
\end{Highlighting}
\end{Shaded}

\begin{verbatim}
## [1] "M;28;81.3"      "male;45;"       "Female;17;57,2" "fem.;64;62.8"
\end{verbatim}

\begin{Shaded}
\begin{Highlighting}[]
\CommentTok{\#Extract the date from the first comment line.}

\NormalTok{date }\OtherTok{\textless{}{-}} \FunctionTok{gsub}\NormalTok{(}\StringTok{"\^{}// Survey data. Created : "}\NormalTok{, }\StringTok{""}\NormalTok{, comments[}\DecValTok{1}\NormalTok{])}
\NormalTok{date}
\end{Highlighting}
\end{Shaded}

\begin{verbatim}
## [1] "21 May 2013"
\end{verbatim}

\#a. Split the character vectors in the vector containing data lines by
semicolon (;) using strsplit.

\begin{Shaded}
\begin{Highlighting}[]
\NormalTok{split\_data }\OtherTok{\textless{}{-}} \FunctionTok{strsplit}\NormalTok{(data\_lines, }\StringTok{";"}\NormalTok{)}
\NormalTok{split\_data}
\end{Highlighting}
\end{Shaded}

\begin{verbatim}
## [[1]]
## [1] "M"    "28"   "81.3"
## 
## [[2]]
## [1] "male" "45"  
## 
## [[3]]
## [1] "Female" "17"     "57,2"  
## 
## [[4]]
## [1] "fem." "64"   "62.8"
\end{verbatim}

\#Find the maximum number of fields retrieved by split. Append rows that
are shorter with NA's.

\begin{Shaded}
\begin{Highlighting}[]
\NormalTok{max\_fields }\OtherTok{\textless{}{-}} \FunctionTok{max}\NormalTok{(}\FunctionTok{sapply}\NormalTok{(split\_data, length))}
\NormalTok{max\_fields}
\end{Highlighting}
\end{Shaded}

\begin{verbatim}
## [1] 3
\end{verbatim}

\begin{Shaded}
\begin{Highlighting}[]
\NormalTok{split\_data }\OtherTok{\textless{}{-}} \FunctionTok{lapply}\NormalTok{(split\_data, }\ControlFlowTok{function}\NormalTok{(x) }\FunctionTok{c}\NormalTok{(x, }\FunctionTok{rep}\NormalTok{(}\ConstantTok{NA}\NormalTok{, max\_fields }\SpecialCharTok{{-}} \FunctionTok{length}\NormalTok{(x))))}
\NormalTok{split\_data}
\end{Highlighting}
\end{Shaded}

\begin{verbatim}
## [[1]]
## [1] "M"    "28"   "81.3"
## 
## [[2]]
## [1] "male" "45"   NA    
## 
## [[3]]
## [1] "Female" "17"     "57,2"  
## 
## [[4]]
## [1] "fem." "64"   "62.8"
\end{verbatim}

\#Use unlist and matrix to transform the data to row-column format.

\begin{Shaded}
\begin{Highlighting}[]
\NormalTok{data\_matrix }\OtherTok{\textless{}{-}} \FunctionTok{matrix}\NormalTok{(}\FunctionTok{unlist}\NormalTok{(split\_data), }\AttributeTok{ncol =}\NormalTok{ max\_fields, }\AttributeTok{byrow =} \ConstantTok{TRUE}\NormalTok{)}
\NormalTok{data\_matrix}
\end{Highlighting}
\end{Shaded}

\begin{verbatim}
##      [,1]     [,2] [,3]  
## [1,] "M"      "28" "81.3"
## [2,] "male"   "45" NA    
## [3,] "Female" "17" "57,2"
## [4,] "fem."   "64" "62.8"
\end{verbatim}

\#From comment lines 2-4, extract the names of the fields. Set these as
colnames for the matrix you just created.

\begin{Shaded}
\begin{Highlighting}[]
\NormalTok{fieldNames }\OtherTok{\textless{}{-}} \FunctionTok{gsub}\NormalTok{(}\StringTok{"\^{}// Field [0{-}9]+: "}\NormalTok{, }\StringTok{""}\NormalTok{, comments[}\DecValTok{2}\SpecialCharTok{:}\DecValTok{4}\NormalTok{])}
\NormalTok{fieldNames}
\end{Highlighting}
\end{Shaded}

\begin{verbatim}
## [1] "Gender"         "Age (in years)" "Weight (in kg)"
\end{verbatim}

\begin{Shaded}
\begin{Highlighting}[]
\FunctionTok{colnames}\NormalTok{(data\_matrix) }\OtherTok{\textless{}{-}}\NormalTok{ fieldNames}
\FunctionTok{colnames}\NormalTok{(data\_matrix)}
\end{Highlighting}
\end{Shaded}

\begin{verbatim}
## [1] "Gender"         "Age (in years)" "Weight (in kg)"
\end{verbatim}

\end{document}
